% Documento Latex com o artigo que estamos escrevendo

% Cabeçalho
% Onde a gente configura o documento
%%%%%%%%%%%%%%%%%%%%%%%%%%%%%%%%%%%%%%%%%%%%%%%%%%%%%%%%%%%%%%%%
\documentclass{article}

\usepackage[brazil]{babel}
\usepackage{graphicx}
\usepackage[round,authoryear,sort]{natbib}

\newcommand{\Title}{Análise de variação de temperatura dos últimos cinco anos}

\input{paises.tex}
% Corpo
% Onde a gente escreve o texto
%%%%%%%%%%%%%%%%%%%%%%%%%%%%%%%%%%%%%%%%%%%%%%%%%%%%%%%%%%%%%%%%
\begin{document}

\title{\Title}
\author{Daniella Motta}

\maketitle
\begin{abstract}
    Meu artigo bem legal.
\end{abstract}


\section{Introdução}
Isso vai ser minha Introdução.
Outra frase da Introdução

Esse já será outro parágrafo.

Trabalho anteriores bem legais fizeram coisas parecidas.
\citep{Hansen2010}.
Isso foi analisado primerio por \citet{Hansen2010}

\section{Metodologia}
\label{sec:metodos}

Aqui eu vou descrever tudo que eu fiz.
Ajustamos uma reta aos cinco últimos anod dos dados
de temperatura média mensal para cada país.
Assim calculamos a taxa de variação da temperatura recente.

A equação da reta é

\begin{equation}
T(t) = a t + b,
\label{eq:reta}
\end{equation}

\noindent
onde  $T$ é a temperatura, $t$ é o tempo, $a$ é o coeficiente angular e $b$ é o coeficiente linear.

Utilizamos a \ref{eq:reta} em um código Python para fazer o ajuste da reta com o método dos mínimos quadrados.
Isso está descrito na seção \ref{sec:metodos}.

\section{Resultados}

Analisamos os dados de 225 países. 
Os países analisados foram \Paises.

\begin{figure} [!tb]
    \centering
    \includegraphics{../figuras/variacao_temperatura.png}
    \caption{
        Variação de temperatura média mensal dos cinco últimos anos.
        a) Países com as cinco maiores variações de temperatura.
        b) Países com as cinco menores variações de temperatura.
    }
    \label{fig:variacao}

\end{figure}
Os resultados da análise de variação de temperatura estão na figura \ref{fig:variacao}.

\bibliographystyle{apalike}
\bibliography{referencias.bib}


\end{document}